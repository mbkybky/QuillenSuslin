\documentclass[12pt]{amsart}
\usepackage{amssymb}

\setlength{\headheight}{8pt}
\setlength{\textheight}{22.4cm}
\setlength{\textwidth}{14.5cm}
\setlength{\oddsidemargin}{.1cm}
\setlength{\evensidemargin}{.1cm}
\setlength{\topmargin}{0.2cm}

\setcounter{secnumdepth}{2}
\setcounter{tocdepth}{1}

\numberwithin{equation}{section}

\usepackage{setspace}
\usepackage{mathrsfs}
\usepackage{enumerate}

\usepackage[colorlinks=true]{hyperref}

\newtheorem{theorem}{Theorem}
\newtheorem{proposition}[theorem]{Proposition}
\newtheorem{corollary}[theorem]{Corollary}
\newtheorem{lemma}[theorem]{Lemma}
\newtheorem{definition}[theorem]{Definition}

\begin{document}

\title{The Proof of the Quillen-Suslin Theorem}
\maketitle

\noindent \url{https://amathew.wordpress.com/2012/01/16/the-quillen-suslin-theorem}.

\section{Unimodular Vector}

\begin{definition}
	Let $A$ be any ring. A vector ${v} \in A^s$ is unimodular if its components generate the unit ideal in $A$. For two unimodular vectors ${v}, {w}$, we write
	$$\displaystyle {v} \sim {w}$$
	if there is a matrix $M \in \mathrm{GL}_s(A)$ such that $M {v} = {w}$. This is clearly an equivalence relation.
\end{definition}

\begin{proposition}\label{prop:6}
	Over a principal ideal domain $R$, any two unimodular vectors are equivalent.
\end{proposition}

\begin{proof}
	In fact, unimodular vectors $v \in R^m$ correspond to imbeddings $R \rightarrow R^m$ which are split injections. But if we have a split injection in this way, the cokernel is free (as we are over a PID), and consequently there is a basis for $R^m$ one of whose elements is $v$. This implies that $v$ is conjugate to $e_1 = (1, 0, \dots, 0)$.
\end{proof}

\begin{theorem}[Horrocks]\label{thm:8}
	Let $A = R[x]$ for $(R, \mathfrak{m})$ a local ring. Then any unimodular vector in $A^s$ one of whose elements has leading coefficient one is equivalent to $e_1$.
\end{theorem}

\begin{proof}
	Let $v(x) = (v_1(x), \dots, v_s(x))$ be a unimodular vector. Suppose without loss of generality that the leading coefficient of $v_1(x)$ is one, so that $v_1 (x) = x^d + a_1 x^{d-1} + \dots$. If $d = 0$, then $v_1$ is a unit and there is nothing to prove. We will induct on $d$.
	
	Then, by making elementary row operations (which don't change the equivalence class of $v(x)$), we can assume that $v_2(x), \dots, v_s(x)$ all have degree $\leq d-1$. Consider the coefficients of these elements. At least one of them must be a unit. In fact, if we reduce mod $\mathfrak{m}$, then not all the $v_i, i \geq 2$ can go to zero or the $v_i(x)$ would not generate the unit ideal mod $\mathfrak{m}$. So let us assume that $v_2(x)$ contains a unit among its coefficients.
	
	The claim is now that we can make elementary row operations so as to find another unimodular vector, in the same equivalence class, one of whose elements is monic of degree $\leq d-1$. If we can show this, then induction on $d$ will easily complete the proof.
	
	Now, here is a lemma: If we have two polynomials $a(x), b(x) \in R[x]$, with $\deg a = d$ and $a$ monic, and $b$ of degree $\leq d-1$ containing at least one coefficient which is a unit, there is a polynomial $a(x) e(x) + b(x) f(x) \in (a(x), b(x))$ of degree $\leq d-1$ whose leading coefficient is one. This is easy to see with a bit of explicit manipulation.
	
	This means that there are $e(x), f(x)$, such that $e(x) v_1(x) + f(x) v_2(x)$ has degree $\leq d-1$ and leading coefficient a unit. If we keep this fact in mind, we can, using row and column operations, modify the vector $v(x)$ such that it contains a monic element of degree $\leq d-1$. We just add appropriate multiples of $v_1, v_2$ to $v_3$ to make the leading coefficient a unit. This works if $s \geq 3$. If $s =1$ or $s= 2$, the lemma can be checked directly.
\end{proof}

\begin{corollary}\label{cor:9}
	If $R$ is local and $v(x) \in R[x]^s$ is a unimodular vector one of whose elements is monic, then $v(x) \sim v(0)$.
\end{corollary}

\begin{proof}
	In fact, $v(0)$ is a unimodular vector in $R$, hence equivalent to $e_1$. We have also seen [in Theorem \ref{thm:8}] that $v(x)$ is equivalent to $e_1$.
\end{proof}

\begin{lemma}\label{lem:10}
	Suppose $v(x) \sim v(0)$ over the localization $R_S[x]$. Then there exists a $c \in S$ such that $v(x) \sim v(x + cy)$ over $R[x, y]$.
\end{lemma}

\begin{proof}
	As before, we can choose a matrix $M(x) \in \mathrm{GL}_s(R_S[x])$ such that $M(x) v(x) = v(0)$, and then the matrix $N(x,y) := M(x)^{-1}M(x+y)$ has the property that
	\[ \displaystyle N(x,y) v(x+y) = v(x). \]
	It follows that if we substitute $cy$ for $y$, then we have
	\[ \displaystyle N(x,cy) v(x+cy) = v(x). \]
	The claim is that we can choose $c \in S$ such that $N(x,cy)$ actually has $R$-coefficients. In fact, this is because $N(x, 0) = I$, which implies that $N(x,y) = I + y W$ for some matrix $W$ with values in $R_S[x,y]$. If we replace $y$ with $cy$ for $c$ an element of $S$, then we can clear the denominators in $W$ and arrange it so that $N(x,cy) \in R[x, y]$.
\end{proof}

\begin{corollary}\label{cor:11}
	Suppose $R$ is any ring, and $v(x) \in R[x]^s$ is a unimodular vector one of whose leading coefficients is one. Then $v(x) \sim v(0)$.
\end{corollary}

\begin{proof}
	Let us consider the set $I$ of $q \in R$ such that $v(x+qy) \sim v(x)$ in $R[x, y]$. If we can show that $1 \in I$, then we will be done, because after applying the homomorphism $x \mapsto 0, R[x, y] \rightarrow R[y]$, we will get that $v(y) \sim v(0)$ in $R[y]$.
	
	We start by observing that $I$ is an ideal. In fact, suppose $v(x+qy) \sim v(x)$ and $v(x + q'y) \sim v(x)$. Then, substituting $x \mapsto x + q'y$ in the first leads to
	\[ \displaystyle v(x + q'y + qy) \sim v(x + q'y) \in R[x,y] \]
	and since $v(x + q'y) \sim v(x)$, we get easily by transitivity that $q + q' \in I$. Similarly, we have to observe that if $q \in I$ and $r \in R$, then $v(x+qry) \sim v(x)$. But this is true because one can substitute $y \mapsto ry$.
	
	Since $I$ is an ideal, to show that $1 \in I$ we just need to show that $I$ is contained in no maximal ideal. Let $\mathfrak{m} \subset R$ be a maximal ideal. We then note that, by what we have already done for local rings [Corollary \ref{cor:9}], we have that
	\[ \displaystyle v(x) \sim v(0 ) \quad \text{in} \quad R_{\mathfrak{m}}[x]. \]
	By Lemma \ref{lem:10}, this means that there is a $q \in R - \mathfrak{m}$ such that $v(x+qy) \sim v(0)$; this means that $q \in I$. So $I$ cannot be contained in $\mathfrak{m}$. Since this applies to any maximal ideal $\mathfrak{m}$, it follows that $I$ must be the unit ideal.
\end{proof}

\begin{theorem}\label{thm:12}
	Let $R = k[x_1, \dots, x_n]$ be a polynomial ring over a principal ideal domain $k$, and let $v \in R^n$ be a unimodular vector. Then $v \sim e_1$.
\end{theorem}

\begin{proof}
	We can now prove this by induction on $n$. When $n = 0$, it is immediate [by Proposition \ref{prop:6}].
	
	Suppose $n \geq 1$. Then we can treat $R$ as $k[x_1, \dots, x_{n-1}, X]$, where we replace $x_n$ by $X$ to make it stand out. We can think of $v = v(X)$ as a vector of polynomials in $X$ with coefficients in the smaller ring $k[x_1, \dots, x_{n-1}]$.
	
	If $v(X)$ has a term with leading coefficient one, then the previous results [Corollary \ref{cor:11}] enable us to conclude that $v(X) \sim v(0)$, and as $v(0)$ lies in $k[x_1, \dots, x_{n-1}]$ we can use induction to work downwards. The claim is that, possibly after a change of variables $x_1, \dots, x_n$, we can always arrange it so that the leading coefficient in $X = x_n$ is one. The relevant change of variables leaves $X = x_n$ constant and
	\[ \displaystyle x_i \mapsto x_i - X^{M^i}, \quad M \gg 0 \quad (1 \leq i < n). \]
	If $M$ is chosen very large, one makes by this substitution the leading term of each of the elements of $v$ a unit. So, without loss of generality we can assume that this is already the case. Thus, we can apply the inductive hypothesis on $n$ to complete the proof.
\end{proof}

\section{Stably Free}


\begin{definition}
	A finitely generated module $P$ over a commutative ring $R$ is said to be stably free if there exists a finitely generated free module $F$ such that the direct sum $P \oplus F$ is a free module.
\end{definition}

\begin{lemma}\label{exact_seq}
	Let $R$ be a ring. Let $0 \to P' \to P \to P'' \to 0$ be a short
	exact sequence of finite projective $R$-modules. If $2$ out of $3$
	of these modules are stably free, then so is the third.
\end{lemma}

\begin{proof}
	Since the modules are projective, the sequence is split. Thus we can
	choose an isomorphism $P = P' \oplus P''$. If $P' \oplus R^{\oplus n}$
	and $P'' \oplus R^{\oplus m}$ are free, then we see that
	$P \oplus R^{\oplus n + m}$ is free. Suppose that $P'$ and $P$ are
	stably free, say $P \oplus R^{\oplus n}$ is free and $P' \oplus R^{\oplus m}$
	is free. Then
	$$
	P'' \oplus (P' \oplus R^{\oplus m}) \oplus R^{\oplus n} =
	(P'' \oplus P') \oplus R^{\oplus m} \oplus R^{\oplus n} =
	(P \oplus R^{\oplus n}) \oplus R^{\oplus m}
	$$
	is free. Thus $P''$ is stably free. By symmetry we get the last of the
	three cases.
\end{proof}

%\begin{proposition}
%	Let $M$ be an $A$-module. Then $M$ has a finite free resolution of length $n \geqq 1$ if and only if $M$ has a stably free resolution of length $n$.
%\end{proposition}
%
%\begin{proof}
%	One direction is trivial, so we suppose given a stably free resolution with the above notation. Let $0 \leqq i < n$ be some integer, and let $F_i, F_{i+1}$ be finite free such that $E_i \oplus F_i$ and $E_{i+1} \oplus F_{i+1}$ are free. Let $F = F_i \oplus F_{i+1}$. Then we can form an exact sequence
%	$$ 0 \to E_n \to \dots \to E_{i+1} \oplus F \to E_i \oplus F \to \dots \to E_0 \to M \to 0 $$
%	in the obvious manner. In this way, we have changed two consecutive modules in the resolution to make them free. Proceeding by induction, we can then make $E_0, E_1$ free, then $E_1, E_2$ free, and so on to conclude the proof of the proposition.
%\end{proof}

\begin{theorem}
	Let $R$ be a noetherian ring such that every finitely generated projective module over $R$ is stably free. Then the same property holds true for $R[x]$.
\end{theorem}

\begin{proof}
	Let $M$ be a finite $R[x]$-module. $M$ has a finite filtration
	$$M = M_0 \supset M_1 \supset \dots \supset M_n = 0$$
	such that each factor $M_i/M_{i+1}$ is isomorphic to $R[x]/P_i$ for some prime $P_i$. In light of Lemma \ref{exact_seq}, it suffices to prove the theorem in case $M = R[x]/P$ where $P$ is prime, which we now assume. In light of the exact sequence
	$$0 \to P \to R[x] \to R[x]/P \to 0$$
	and Lemma \ref{exact_seq}, we note that $M$ is stably free if and only if $P$ does.
	
	Let $\mathfrak{p} = P \cap R$. Then $\mathfrak{p}$ is prime in $R$. Suppose there is some $M = R[x]/P$ which does not admit a finite free resolution. Among all such $M$ we select one for which the intersection $\mathfrak{p}$ is maximal in the family of prime ideals obtained as above. This is possible in light of one of the basic properties characterizing Noetherian rings.
	
	Let $R_0 = R/\mathfrak{p}$ so $R_0$ is entire. Let $P_0 = P/\mathfrak{p}R[x]$. Then we may view $M$ as an $R_0[x]$-module, equal to $R_0/P_0$. Let $f_1, \dots, f_n$ be a finite set of generators for $P_0$, and let $f$ be a polynomial of minimal degree in $P_0$. Let $K_0$ be the quotient field of $R_0$. By the euclidean algorithm, we can write
	$$f_i = q_i f + r_i \quad \text{for } i = 1, \dots, n$$
	with $q_i, r_i \in K_0[x]$ and $\deg r_i < \deg f$. Let $d_0$ be a common denominator for the coefficients of all $q_i, r_i$. Then $d_0 \neq 0$ and
	$$d_0 f_i = q'_i f + r'_i$$
	where $q'_i = d_0 q_i$ and $r'_i = d_0 r_i$ lie in $R_0[x]$. Since $\deg f$ is minimal in $P_0$ it follows that $r'_i = 0$ for all $i$, so
	$$d_0 P_0 \subset R_0[x]f = (f).$$
	
	Let $N_0 = P_0/(f)$, so $N_0$ is a module over $R_0[x]$, and we can also view $N_0$ as a module over $R[x]$. When so viewed, we denote $N_0$ by $N$. Let $d \in R$ be any element reducing to $d_0 \pmod{\mathfrak{p}}$. Then $d \notin \mathfrak{p}$ since $d_0 \neq 0$. The module $N_0$ has a finite filtration such that each factor module of the filtration is isomorphic to some $R_0[x]/\bar{Q}_0$ where $\bar{Q}_0$ is an associated prime of $N_0$. Let $Q$ be the inverse image of $\bar{Q}_0$ in $R[x]$. These prime ideals $Q$ are precisely the associated primes of $N$ in $R[x]$. Since $d_0$ kills $N_0$ it follows that $d$ kills $N$ and therefore $d$ lies in every associated prime of $N$. By the maximality property in the selection of $P$, it follows that every one of the factor modules in the filtration of $N$ has a finite free resolution, and by Lemma \ref{exact_seq} it follows that $N$ itself has a finite free resolution.
	
	Now we view $R_0[x]$ as an $R[x]$-module, via the canonical homomorphism
	$$R[x] \to R_0[x] = R[x]/\mathfrak{p}R[x].$$
	By assumption, $\mathfrak{p}$ is stably free as $R$-module, then $\mathfrak{p}[x] = \mathfrak{p}R[x]$ is stably free as $R[x]$-module. From the exact sequence
	$$0 \to \mathfrak{p}R[x] \to R[x] \to R_0[x] \to 0$$
	we conclude that $R_0[x]$ is stably free as $R[x]$-module.
	
	Since $R_0$ is entire, it follows that the principal ideal $(f)$ in $R_0[x]$ is $R[x]$-isomorphic to $R_0[x]$, and therefore is stably free as $R[x]$-module. Lemma \ref{exact_seq} applied to the exact sequence of $R[x]$-modules
	$$0 \to (f) \to P_0 \to N \to 0$$
	shows that $P_0$ has a finite free resolution; and further applied to the exact sequence
	$$0 \to \mathfrak{p}R[x] \to P \to P_0 \to 0$$
	shows that $P$ has a finite free resolution, thereby concluding the proof.
\end{proof}

By induction, we see:

\begin{corollary}\label{cor:3}
	Every finitely generated projective module over $k[x_1, \dots, x_n]$, for any field $k$, is necessarily stably free.
\end{corollary}

\section{Quillen-Suslin theorem}

\begin{theorem}[Quillen-Suslin]\label{thm:13}
	A finitely generated projective module over $k[x_1, \dots, x_n]$ for $k$ a principal ideal domain is free.
\end{theorem}

\begin{proof}
	 By Corollary \ref{cor:3}, we only need to show that a stably free module over $R = k[x_1, \dots, x_n]$ is fre. That is, if $P$ is such a finitely generated module such that $P \oplus R^m \simeq R^{m'}$, then $P$ is free. By induction on $m$, one reduces to the case $m = 1$. In this case we have an exact sequence
	\[ \displaystyle 0 \rightarrow R \rightarrow R^{m'} \rightarrow P \rightarrow 0 \]
	and we have to conclude that the cokernel $P$ is free.
	
	But the injection $R \rightarrow R^{m'}$ corresponds to a unimodular vector, and we have seen [by Theorem \ref{thm:12}] that this is isomorphic to the standard embedding $e_1: R \rightarrow R^{m'}$, whose cokernel is obviously free. Thus $P$ is free.
\end{proof}

\end{document}